\documentclass[border=5mm]{standalone}
\usepackage{luatex85}
\usepackage{luamplib}
\begin{document}
\mplibtextextlabel{enable}
\mplibnumbersystem{binary}
\begin{mplibcode}
beginfig(1);

numeric pi;
% approximate value
pi := 3.14159;  show pi;
% up to 33 digits of precision 
pi := 3.14159265358979323846264338327950; show pi;
% using some geometry...
pi := 1/4 arclength (quartercircle scaled 16); show pi;
% as many digits as are needed...
vardef getpi =
  numeric lasts, t, s, n, na, d, da;
  lasts=0; s=t=3; n=1; na=0; d=0; da=24;
  forever:
    exitif lasts=s;
    lasts := s;
    n := n+na; na := na+8;
    d := d+da; da := da+32; 
    t := t*n/d;
    s := s+t;
  endfor
  s
enddef;
pi := getpi;  show pi;



endfig;
\end{mplibcode}
\end{document}

